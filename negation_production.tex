\documentclass[man,floatsintext,draftall]{apa6}
\usepackage{lmodern}
\usepackage{amssymb,amsmath}
\usepackage{ifxetex,ifluatex}
\usepackage{fixltx2e} % provides \textsubscript
\ifnum 0\ifxetex 1\fi\ifluatex 1\fi=0 % if pdftex
  \usepackage[T1]{fontenc}
  \usepackage[utf8]{inputenc}
\else % if luatex or xelatex
  \ifxetex
    \usepackage{mathspec}
  \else
    \usepackage{fontspec}
  \fi
  \defaultfontfeatures{Ligatures=TeX,Scale=MatchLowercase}
\fi
% use upquote if available, for straight quotes in verbatim environments
\IfFileExists{upquote.sty}{\usepackage{upquote}}{}
% use microtype if available
\IfFileExists{microtype.sty}{%
\usepackage{microtype}
\UseMicrotypeSet[protrusion]{basicmath} % disable protrusion for tt fonts
}{}
\usepackage{hyperref}
\hypersetup{unicode=true,
            pdftitle={The production of negation in parents' and children's speech},
            pdfauthor={First Author~\& Ernst-August Doelle},
            pdfkeywords={keywords},
            pdfborder={0 0 0},
            breaklinks=true}
\urlstyle{same}  % don't use monospace font for urls
\usepackage{graphicx,grffile}
\makeatletter
\def\maxwidth{\ifdim\Gin@nat@width>\linewidth\linewidth\else\Gin@nat@width\fi}
\def\maxheight{\ifdim\Gin@nat@height>\textheight\textheight\else\Gin@nat@height\fi}
\makeatother
% Scale images if necessary, so that they will not overflow the page
% margins by default, and it is still possible to overwrite the defaults
% using explicit options in \includegraphics[width, height, ...]{}
\setkeys{Gin}{width=\maxwidth,height=\maxheight,keepaspectratio}
\IfFileExists{parskip.sty}{%
\usepackage{parskip}
}{% else
\setlength{\parindent}{0pt}
\setlength{\parskip}{6pt plus 2pt minus 1pt}
}
\setlength{\emergencystretch}{3em}  % prevent overfull lines
\providecommand{\tightlist}{%
  \setlength{\itemsep}{0pt}\setlength{\parskip}{0pt}}
\setcounter{secnumdepth}{0}
% Redefines (sub)paragraphs to behave more like sections
\ifx\paragraph\undefined\else
\let\oldparagraph\paragraph
\renewcommand{\paragraph}[1]{\oldparagraph{#1}\mbox{}}
\fi
\ifx\subparagraph\undefined\else
\let\oldsubparagraph\subparagraph
\renewcommand{\subparagraph}[1]{\oldsubparagraph{#1}\mbox{}}
\fi

%%% Use protect on footnotes to avoid problems with footnotes in titles
\let\rmarkdownfootnote\footnote%
\def\footnote{\protect\rmarkdownfootnote}


  \title{The production of negation in parents' and children's speech}
    \author{First Author\textsuperscript{1}~\& Ernst-August Doelle\textsuperscript{1,2}}
    \date{}
  
\shorttitle{Child and Parent Production of Negation}
\affiliation{
\vspace{0.5cm}
\textsuperscript{1} Wilhelm-Wundt-University\\\textsuperscript{2} Konstanz Business School}
\keywords{keywords\newline\indent Word count: X}
\usepackage{csquotes}
\usepackage{upgreek}
\captionsetup{font=singlespacing,justification=justified}

\usepackage{longtable}
\usepackage{lscape}
\usepackage{multirow}
\usepackage{tabularx}
\usepackage[flushleft]{threeparttable}
\usepackage{threeparttablex}

\newenvironment{lltable}{\begin{landscape}\begin{center}\begin{ThreePartTable}}{\end{ThreePartTable}\end{center}\end{landscape}}

\makeatletter
\newcommand\LastLTentrywidth{1em}
\newlength\longtablewidth
\setlength{\longtablewidth}{1in}
\newcommand{\getlongtablewidth}{\begingroup \ifcsname LT@\roman{LT@tables}\endcsname \global\longtablewidth=0pt \renewcommand{\LT@entry}[2]{\global\advance\longtablewidth by ##2\relax\gdef\LastLTentrywidth{##2}}\@nameuse{LT@\roman{LT@tables}} \fi \endgroup}


\usepackage{lineno}

\linenumbers

\authornote{Add complete departmental affiliations for each author here. Each new line herein must be indented, like this line.

Enter author note here.

Correspondence concerning this article should be addressed to First Author, Postal address. E-mail: \href{mailto:my@email.com}{\nolinkurl{my@email.com}}}

\abstract{
this is the abstract


}

\begin{document}
\maketitle

\hypertarget{introduction}{%
\section{Introduction}\label{introduction}}

Study 1 Questions

\begin{itemize}
\item
  What is the overall trajectory of negative forms in child production?

  \begin{itemize}
  \tightlist
  \item
    Does the development of negation follow a no \textgreater{} not \textgreater{} nt cline? (Cameron-faulkner et al)
  \end{itemize}
\item
  Do positive variants of the negative constructions exist too?

  \begin{itemize}
  \tightlist
  \item
    Are early \enquote{can't} and \enquote{don't} examples unanalyzed wholes? (Klima \& Bellugi 1966; Bloom 1970) Do children produce \enquote{can't} and \enquote{don't} before using \enquote{do} and \enquote{can}?
  \end{itemize}
\item
  Proportion of no vs.~not vs.~nt broken down by mean length of utterance

  \begin{itemize}
  \tightlist
  \item
    instead of age, put mean length of utterance on the x axis?
  \end{itemize}
\end{itemize}

Study 2 Questions:

\begin{itemize}
\item
  What are early constructions in parents' and children's speech?
\item
  Do children's early negative utterances differ so much from those used by adults? (Thornton \& Tesan 2013)

  \begin{itemize}
  \item
    How common are ungrammatical non-adult like combinations?

    \begin{itemize}
    \item
      How many pre-sentential negation? (NEG + Subject + Predicate)

      \begin{itemize}
      \tightlist
      \item
        How many sentence internal? (Subj + NEG + Predicate)
      \end{itemize}
    \item
      Is negation external at the beginning? (appear before subjects) Does a NEG + S schema mark the beginning of negation? (McNeill \& McNeill)
    \item
      How many are optional infinitive: it not fit in here, it don't fit in here?
    \end{itemize}
  \item
    control MLU: which forms are common among 1/2/3/\ldots{} word utterances?
  \item
    exclude single \enquote{no} (as well as anaphoric no) utterances from \enquote{no + more words}
  \end{itemize}
\item
  What is anaphoric negation negating?
\item
  How productive are early forms of negation?

  \begin{itemize}
  \tightlist
  \item
    average neg + \#WORD per child as measure of productivity
  \end{itemize}
\end{itemize}

\hypertarget{previous-studies}{%
\subsection{Previous Studies}\label{previous-studies}}

Formal and functional development of negation

\begin{enumerate}
\def\labelenumi{\arabic{enumi}.}
\item
  Klima \& Bellugi. 1966. Syntactic regularities in the speech of children. In Psycholinguistic papers, ed. J. Lyons and R. Wales, 183-208. Edinburgh: Edinburgh University Press.
\item
  Bellugi (1967). The acquisition of negation. Doctoral dissertation, Harvard University, Cambridge, Mass.
\item
  McNeill \& McNeill 1968: Japanese
\item
  Bloom, L. (1970). Language development: Form and function in emerging grammars. Cambridge, MA: MIT Press.
\item
  Lord (1974): Variations in the pattern of acquisition of negation
\item
  Wode, H. (1977). Four early stages in the development of L1 negation. Journal of Child Language 4, 87--102.
\item
  Pea (1978): the development of negation in early child language. dissertation
\item
  De Villiers, P., and J. G. De Villiers (1979) \enquote{Form and function in the development of sentence negation}, Papers and Reports on Child Language Development, 17, 57- 64.
\item
  Pea Dissertation
\item
  Clahsen, Harald. 1983. Some remarks on the acquisition of German negation. Journal of Child Language 10:465-469.
\item
  Choi, S. (1988). The semantic development of negation: A cross-linguistic longitudinal study. Journal of Child Language, 15, 517--531.
\item
  Weissenborn, Juirgen, and Monica Verrips. 1989. Negation as a window to the structure of early child language. Ms., Max Planck Institut fur Psycholinguistik, Nijmegen.
\item
  Deprez, Viviane and Amy Pierce. 1993. Negation and Functional Projections in Early Grammar. Linguistic Inquiry 24, no. 1: 25-67.
\item
  Stromswold, K. (1997) The Acquisition of Inversion and Negation in English: A Reply to Deprez and Pierce', ms. Rutgers.
\item
  Drozd (1995): Child English pre-sentential negation as metalinguistic exclamatory sentence negation. JCL
\item
  Hamann 2000
\item
  Cameron-Faulkner, T., Lieven, E., \& Theakston, A. (2007). What part of no do children not understand? A usage-based account of multiword negation. Journal of Child Language, 34, 251--282.
\item
  Guidetti (2000): Pragmatic study of agreement and refusal messages in young French children. Journal of Pragmatics
\item
  Guidetti (2005): Yes or no? How young French children combinegestures and speech to agree and refuse. JCL
\item
  Schutze (2010) The Status of Nonagreeing Don't and Theories of Root Infinitives
\item
  Dimroth (2010): The Acquisition of Negation
\item
  Thornton \& Tesan (2013): sentential negation in early child English
\item
  Nordmeyer \& Frank (2014): Individual variation in children's early production of negation
\end{enumerate}

\hypertarget{current-study}{%
\subsection{Current Study}\label{current-study}}

Acquisition of negation should concern itself with two notions: 1. negative morpheme 2. compositional complexity. By negative morpheme, we mean the kinds of morphemes that at each stage of acquisition are mapped to negative meanings. English has adverbal and adnominal mophemes that encode the concept of negation. We can look at how each form-meaning mapping emerges in children's development. Second by compositional complexity, we mean the types of elements that each morphemes succesfully negates at each stage of development. Under stuch analysis negation may have been successfully acquired to operate on locative elements but not identity relations. Compositional complexity of negation at each stage also helps us understand how quickly children generalize the function of negation beyond specific arguments it takes in the child's input.

\hypertarget{study-1-large-scale-metrics}{%
\section{Study 1: Large-scale metrics}\label{study-1-large-scale-metrics}}

\hypertarget{methods}{%
\subsection{Methods}\label{methods}}

For samples of parents' and children's speech, we used the online database \href{childes-db.stanford.edu}{childes-db} and its associated R programming package \texttt{childesr} (Sanchez et al., 2018). Childes-db is an online interface to the child language components of \href{https://talkbank.org/}{TalkBank}, namely \href{https://childes.talkbank.org/}{CHILDES} (MacWhinney, 2000) and \href{https://phonbank.talkbank.org/}{PhonBank}. Two collections of corpora were selected: English-North America and English-UK.

\hypertarget{procedure}{%
\subsubsection{Procedure}\label{procedure}}

All word tokens were tagged for the following information: 1. The speaker role (parent vs.~child), 2. the age of the child when the word was produced, 3. the type of the utterance the word appeared in (declarative, question, imperative, other)\footnote{This study grouped utterance types into four main categories: \enquote{declarative}, \enquote{question}, \enquote{imperative}, and \enquote{other}. Utterance type categorization followed the convention used in the \href{https://talkbank.org/manuals/CHAT.html\#_Toc486414422}{TalkBank manual}. The utterance types are similar to sentence types (declarative, interrogative, imperative) with one exception: the category \enquote{question} consists of interrogatives as well as rising declaratives (i.e.~declaratives with rising question intonation). In the transcripts, declaratives are marked with a period, questions with a question mark, and imperatives with an exclamation mark. It is important to note that the manual also provides \href{https://talkbank.org/manuals/CHAT.html\#_Toc486414431}{terminators for special-type utterances}. Among the special type utterances, this study included the following in the category \enquote{questions}: trailing off of a question, question with exclamation, interruption of a question, and self-interrupted question. The category imperatives also included \enquote{emphatic imperatives}. The rest of the special type utterances such as \enquote{interruptions} and \enquote{trailing off} were included in the category \enquote{other}.}, 4. whether the word was positive or negative, and 5. the type of negative word produced. For this study we considerd the following classes of negative morphemes in English: the forms \emph{no} and \emph{not}, all possible negative clitic auxiliary forms with \emph{n't} (i.e. \emph{ain't}, \emph{isn't}, \emph{amn't}, \emph{aren't}, \emph{wasn't}, \emph{weren't}, \emph{don't}, \emph{doesn't}, \emph{didn't}, \emph{won't}, \emph{shan't}, \emph{hasn't}, \emph{havn't}, \emph{hadn't}, \emph{shouldn't}, \emph{can't}, \emph{couldn't}, \emph{may'nt}, \emph{might'nt}, \emph{would'nt}, and \emph{mustn't}) as well as their positive forms without \emph{n't} as controls, negative pronouns (\emph{nothing}, \emph{nobody}, \emph{no-one}, \emph{nowhere}) and their positive existential and universal variants (\emph{something}, \emph{everything}, \emph{somebody}, \emph{everybody}, \emph{someone}, \emph{everyone}, \emph{somewhere}, \emph{everywhere}), negative quantifier \emph{none} and its existential and universal variants (\emph{some}, \emph{all}), the negative adverb of frequency \emph{never} and its existential and universal variants (\emph{sometimes}, \emph{always}), and finally derivational negative forms with morphemes \emph{un-} (e.g.~unhappy), \emph{in-} (e.g.~invisible), \emph{dis-}(e.g.~disappear), \emph{de-} (e.g.~defrost), \emph{non-} (e.g.~nonsense), and \emph{-less} (e.g.~careless).

\hypertarget{exclusion-criteria}{%
\subsubsection{Exclusion Criteria}\label{exclusion-criteria}}

First, unintelligible tokens were excluded (N = 379,549). Second, tokens that had missing information on children's age were excluded (N = 1,060,766). Third, tokens outside the age range of 1 to 6 years were excluded (N = 658,207). The collection contained the speech of 570 children and their parents after the exclusions.

\hypertarget{results}{%
\subsection{Results}\label{results}}

Following Cameron-Faulkner, Lieven, and Theakston (2007), we first look at the proportions of different categories of negation in parents' and children's speech between the ages of 1-6 years. As the right panel on Figure \ref{fig:negationProportionPlot} shows, of all negative froms parents produce, the majority are the contracted auxiliary negation \emph{n't}, followed by \emph{no} and then \emph{not} respectively. Other forms of negation like negative quantification pronouns (e.g. \emph{nothing}) or negative adverbs of frequency (e.g. \emph{never}) are much less frequent. In children's productions and between the ages of 12-18 months, almost all negative forms are instances of \emph{no}, with some contracted auxiliary negatives like \emph{don't} and \emph{can't}. As children grow older, the proportions of \emph{not} and its contracted form \emph{n't} increase while the proportion of \emph{no} decreases. Similar to Cameron-Faulkner et al. (2007) we find that children start productions of \emph{no} earlier than other forms. However, we do not find the full form \emph{not} to be produced before its contracted form \emph{n't}. The results in Figure \ref{fig:negationProportionPlot} suggest that children start producing \emph{not} and \emph{n't} around the same time, if not slightly earlier for \emph{n't}.

\begin{figure}
\centering
\includegraphics{negation_production_files/figure-latex/negationProportionPlot-1.pdf}
\caption{\label{fig:negationProportionPlot}Proportion of different categories of negation in parents' and children's speech between 1 to 6 years of age.}
\end{figure}

Figure \ref{fig:negationRelativeFrequency} shows the relative frequency of the morphemes \emph{no}, \emph{not} and \emph{n't} per thousand words in the speech of parents and children. Children start producing \emph{no} between 12-18 months and they immediately surpass their parents' rate of production for this morpheme. Betwen 18-42 months children produce two to three times more instance of \emph{no} than their parents. This rapid incrase and high frequency of \emph{no} may be partly because parents ask many yes/no questions from children in this age range. After 42 months the frequency of \emph{no} reduces substantially and gets closer to parents' level of 10 per thousand. For the negative morpheme \emph{not}, children start their productions between 12-24 months and by 30 months of age, they are producing \emph{not} at the same rate as their parents (5 per thousand). After 36 months children's rate of \emph{not} productions stay similar to their parents. Finally for the contracted form \emph{n't}, children's productions start between 12-18 months and by 24 months they reach a rate of 5 instances per thousand words. They keep increasing this rate until they reach their parents' rate of 15 instances per thousand at age 36 months. It is important to note that for all these negative forms, children have reached a substantial level of production by 30 months of age.

\begin{figure}
\centering
\includegraphics{negation_production_files/figure-latex/negationRelativeFrequency-1.pdf}
\caption{\label{fig:negationRelativeFrequency}Relative frequency of the response particle \emph{no}, verb phrase negation \emph{not}, and its contracted form \emph{n't}}
\end{figure}

Klima and Bellugi (1966) reported that in their sample, children did not produce the positive auxiliary forms like \emph{can} or \emph{do} even though they were already producing the negative variants like \emph{can't} and \emph{don't}. Based on this, they hypothesized that the negative auxiliaries are learned as unanalyzed chunks. Choi (1988) concurred and added \emph{won't} to the list of early unanalyzed negative chunks. Figure \ref{fig:auxRelFreq} shows the relative frequency of positive and negative auxiliary forms in the speech of children and their parents. Our results show that overall, children start producing the positive and negative auxiliary forms around the same time and they always produce the positive forms at a higher rate than negative ones. Therefore, the claim that negative auxiliary forms are learned before the positive ones is not supported by our data.

\begin{figure}
\centering
\includegraphics{negation_production_files/figure-latex/auxRelFreq-1.pdf}
\caption{\label{fig:auxRelFreq}Relative frequency (per thousand words) of positive auxiliary forms such as \emph{do}, \emph{are}, and \emph{can} as well as their contracted negatives in the speech of parents and children.}
\end{figure}

The auxiliary category in our previous figures lump together a wide variety of auxiliary verbs that develop at different rates. Figure \ref{fig:AuxWords} shows the production of common negative auxiliary verbs in the speech of children and parents, sorted from top-left to bottom-right based on frequency. The most frequent negative auxiliary form in child-directed speech is \emph{don't} and it is also the earliest and most frequent auxiliary form in children's speech. Children start producing it between 12-24 months and they quickly reach the parents' rate at 36 months. Perhaps the fastest development occurs with the auxiliary \emph{can't}. Children start producing it between 18-24 months and very quickly surpass their parents' rate.

\begin{figure}
\centering
\includegraphics{negation_production_files/figure-latex/AuxWords-1.pdf}
\caption{\label{fig:AuxWords}Relative frequency of negated auxiliary verbs in the speech of parents (green triangles) and children (red circles). The dashed line marks 24 months on the x axis.}
\end{figure}

Figure \ref{fig:pronouns} shows the development of negative and positive indefinite pronouns: \emph{everything}, \emph{nothing}, \emph{something}. Children start producing these words quite early as well, with \emph{nothing} reaching the parent level of production at 30 months.

\begin{figure}
\centering
\includegraphics{negation_production_files/figure-latex/pronouns-1.pdf}
\caption{\label{fig:pronouns}Relative frequency of pronouns \emph{everything}, \emph{something}, and \emph{nothing}}
\end{figure}

\begin{figure}
\centering
\includegraphics{negation_production_files/figure-latex/quantifiers-1.pdf}
\caption{\label{fig:quantifiers}Relative frequency of quantifeirs \emph{none}, \emph{some}, and \emph{all}.}
\end{figure}

Adverbs of frequency

\begin{figure}
\centering
\includegraphics{negation_production_files/figure-latex/adverbs-1.pdf}
\caption{\label{fig:adverbs}Relative frequency for adverbs of frequency \emph{always}, \emph{never}, and \emph{sometimes} in the speech of parents and children.}
\end{figure}

\hypertarget{conclusions}{%
\subsection{Conclusions}\label{conclusions}}

Essentially the answers to these questions:

\begin{itemize}
\tightlist
\item
  What is the overall trajectory of negative forms in child production?

  \begin{itemize}
  \tightlist
  \item
    Does the development of negation follow a no \textgreater{} not \textgreater{} nt cline? (Cameron-faulkner et al)
  \item
    How many children are found to produce no/not/nt at each age?
  \end{itemize}
\item
  Do positive variants of the negative constructions exist too?

  \begin{itemize}
  \tightlist
  \item
    Are early \enquote{can't} and \enquote{don't} examples unanalyzed wholes? (Klima \& Bellugi 1966; Bloom 1970) Do children produce \enquote{can't} and \enquote{don't} before using \enquote{do} and \enquote{can}?
  \end{itemize}
\item
  Proportion of no vs.~not vs.~nt broken down by mean length of utterance

  \begin{itemize}
  \tightlist
  \item
    instead of age, put mean length of utterance on the x axis?
  \end{itemize}
\end{itemize}

\hypertarget{study-2-early-productions}{%
\section{Study 2: Early Productions}\label{study-2-early-productions}}

\hypertarget{participants}{%
\subsection{Participants}\label{participants}}

\hypertarget{material}{%
\subsection{Material}\label{material}}

\hypertarget{procedure-1}{%
\subsection{Procedure}\label{procedure-1}}

\hypertarget{data-analysis}{%
\subsection{Data analysis}\label{data-analysis}}

\hypertarget{results-1}{%
\section{Results}\label{results-1}}

\hypertarget{discussion}{%
\section{Discussion}\label{discussion}}

\newpage

\hypertarget{references}{%
\section{References}\label{references}}

\begingroup
\setlength{\parindent}{-0.5in}
\setlength{\leftskip}{0.5in}

\endgroup

\hypertarget{refs}{}
\leavevmode\hypertarget{ref-cameron2007part}{}%
Cameron-Faulkner, T., Lieven, E., \& Theakston, A. (2007). What part of no do children not understand? A usage-based account of multiword negation. \emph{Journal of Child Language}, \emph{34}(2), 251--282.

\leavevmode\hypertarget{ref-choi1988semantic}{}%
Choi, S. (1988). The semantic development of negation: A cross-linguistic longitudinal study. \emph{Journal of Child Language}, \emph{15}(3), 517--531.

\leavevmode\hypertarget{ref-klimaBellugi1966}{}%
Klima, E. S., \& Bellugi, U. (1966). Syntactic regularities in the speech of children. In \emph{Psycholinguistics papers} (pp. 183--207). Edinburgh University Press.

\leavevmode\hypertarget{ref-macwhinney2000childes}{}%
MacWhinney, B. (2000). \emph{The CHILDES project: The database} (Vol. 2). Mahwah, NJ: Erlbaum.

\leavevmode\hypertarget{ref-sanchez2018childes}{}%
Sanchez, A., Meylan, S., Braginsky, M., MacDonald, K., Yurovsky, D., \& Frank, M. C. (2018). Childes-db: A flexible and reproducible interface to the child language data exchange system. PsyArXiv. Retrieved from \url{psyarxiv.com/93mwx}


\end{document}
