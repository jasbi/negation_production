% Template for Cogsci submission with R Markdown

% Stuff changed from original Markdown PLOS Template
\documentclass[10pt, letterpaper]{article}

\usepackage{cogsci}
\usepackage{pslatex}
\usepackage{float}
\usepackage{caption}

% amsmath package, useful for mathematical formulas
\usepackage{amsmath}

% amssymb package, useful for mathematical symbols
\usepackage{amssymb}

% hyperref package, useful for hyperlinks
\usepackage{hyperref}

% graphicx package, useful for including eps and pdf graphics
% include graphics with the command \includegraphics
\usepackage{graphicx}

% Sweave(-like)
\usepackage{fancyvrb}
\DefineVerbatimEnvironment{Sinput}{Verbatim}{fontshape=sl}
\DefineVerbatimEnvironment{Soutput}{Verbatim}{}
\DefineVerbatimEnvironment{Scode}{Verbatim}{fontshape=sl}
\newenvironment{Schunk}{}{}
\DefineVerbatimEnvironment{Code}{Verbatim}{}
\DefineVerbatimEnvironment{CodeInput}{Verbatim}{fontshape=sl}
\DefineVerbatimEnvironment{CodeOutput}{Verbatim}{}
\newenvironment{CodeChunk}{}{}

% cite package, to clean up citations in the main text. Do not remove.
\usepackage{apacite}

% KM added 1/4/18 to allow control of blind submission


\usepackage{color}

% Use doublespacing - comment out for single spacing
%\usepackage{setspace}
%\doublespacing


% % Text layout
% \topmargin 0.0cm
% \oddsidemargin 0.5cm
% \evensidemargin 0.5cm
% \textwidth 16cm
% \textheight 21cm

\title{Combinatorial Capacity of English Negation in Child Language}


\author{{\large \bf Morton Ann Gernsbacher (MAG@Macc.Wisc.Edu)} \\ Department of Psychology, 1202 W. Johnson Street \\ Madison, WI 53706 USA \AND {\large \bf Masoud Jasbi (jasbi@ucdavis.edu)} \\ Department of Linguistics, 469 Kerr Hall, One Shields Avenue \\ Davis, CA 95 USA}


\begin{document}

\maketitle

\begin{abstract}
Negation is very important for langauge and thought. How does it develop
in the language of children? There has been many guessses like
rejection, non-existence, denail, etc. but it has been hard to assess
because these concepts are vaguge. Here we assess the combinatorial
capacity of early negation in children's productions, and use words
negation combines with as a proxy for early concepts expressed by it. We
show some important stuff.

\textbf{Keywords:}
Add your choice of indexing terms or keywords; kindly use a semi-colon;
between each term.
\end{abstract}

\hypertarget{introduction}{%
\section{Introduction}\label{introduction}}

Negation is an abstract concept found in all human languages and crucial
to everyday communication. It can help a coffee shop divide its menu
into ``coffee'' and ``not coffee'' sections, with the ``not coffee''
section bringing together diverse items that otherwise cannot be
labeled. It can help us regulate others' actions in a sign like ``no
mask, no entry''. It also helps us communicate our feelings and desires;
what we do not want or dislike. But how does this crucial abstract
concept emerge in humans? Does language play a role in its emergence or
does language simply adopt it for communication?

There has been several influential hypotheses on the conceptual origin
of negation.

Emotional: rejection (I not like it, not want it)

Perceptual: non-existence (no juice, no more milk, no fish in the
bathroom, I do not have underpants), failure, Locatives (no in there,
daddy was not on the phone), non-events (the dog not barking)

Motor control: prohibition (do not spill milk), inability (I cannot zip
it) 4. Linguistic: labeling and word learning (this is not a bunny, not
red, this isn't a reptile)

Epistemic: know, think, etc. (I not know)

Things that stay outside: why not, not now/today/again (temporal), the
dog not barking (verbs and actions)

In this paper, we address the same topic in a slightly different way. We
start with the widely accepted assumption that negation is a higher
order operator, affecting the meaning of lower level concepts. The
question we ask is: what type of concepts does linguistic negation
operate on in early child language? Do we find negation starting in a
limited conceptual domain and then expanding to others? Or do we find it
operating across different conceptual domains as early as we can attest
it?

Caveat on production vs comprehension.

\hypertarget{study}{%
\section{Study}\label{study}}

First level headings should be in 12 point , initial caps, bold and
centered. Leave one line space above the heading and
1/4\textasciitilde line space below the heading.

\hypertarget{method}{%
\subsection{Method}\label{method}}

Second level headings should be 11 point , initial caps, bold, and flush
left. Leave one line space above the heading and 1/4\textasciitilde{}
line space below the heading.

\hypertarget{third-level-headings}{%
\subsubsection{Third-Level Headings}\label{third-level-headings}}

Third-level headings should be 10 point , initial caps, bold, and flush
left. Leave one line space above the heading, but no space after the
heading.

\hypertarget{discussion}{%
\section{Discussion}\label{discussion}}

Use standard APA citation format. Citations within the text should
include the author's last name and year. If the authors' names are
included in the sentence, place only the year in parentheses, as in
(1972), but otherwise place the entire reference in parentheses with the
authors and year separated by a comma (Newell \& Simon, 1972). List
multiple references alphabetically and separate them by semicolons
(Chalnick \& Billman, 1988; Newell \& Simon, 1972). Use the et.
al.~construction only after listing all the authors to a publication in
an earlier reference and for citations with four or more authors.

For more information on citations in RMarkdown, see
\textbf{\href{http://rmarkdown.rstudio.com/authoring_bibliographies_and_citations.html\#citations}{here}.}

\hypertarget{footnotes}{%
\subsection{Footnotes}\label{footnotes}}

Indicate footnotes with a number\footnote{Sample of the first
footnote.} in the text. Place the footnotes in 9 point type at the
bottom of the page on which they appear. Precede the footnote with a
horizontal rule.\footnote{Sample of the second footnote.} You can also
use markdown formatting to include footnotes using this
syntax.\footnote{Sample of a markdown footnote.}

\hypertarget{figures}{%
\subsection{Figures}\label{figures}}

All artwork must be very dark for purposes of reproduction and should
not be hand drawn. Number figures sequentially, placing the figure
number and caption, in 10 point, after the figure with one line space
above the caption and one line space below it. If necessary, leave extra
white space at the bottom of the page to avoid splitting the figure and
figure caption. You may float figures to the top or bottom of a column,
or set wide figures across both columns.

\hypertarget{two-column-images}{%
\subsection{Two-column images}\label{two-column-images}}

You can read local images using png package for example and plot it like
a regular plot using grid.raster from the grid package. With this method
you have full control of the size of your image. \textbf{Note: Image
must be in .png file format for the readPNG function to work.}

You might want to display a wide figure across both columns. To do this,
you change the \texttt{fig.env} chunk option to \texttt{figure*}. To
align the image in the center of the page, set \texttt{fig.align} option
to \texttt{center}. To format the width of your caption text, you set
the \texttt{num.cols.cap} option to \texttt{2}.

\begin{CodeChunk}
\begin{figure*}[h]

{\centering \includegraphics{figs/2-col-image-1} 

}

\caption[This image spans both columns]{This image spans both columns. And the caption text is limited to 0.8 of the width of the document.}\label{fig:2-col-image}
\end{figure*}
\end{CodeChunk}

\hypertarget{one-column-images}{%
\subsection{One-column images}\label{one-column-images}}

Single column is the default option, but if you want set it explicitly,
set \texttt{fig.env} to \texttt{figure}. Notice that the
\texttt{num.cols} option for the caption width is set to \texttt{1}.

\begin{CodeChunk}
\begin{figure}[H]

{\centering \includegraphics{figs/image-1} 

}

\caption[One column image]{One column image.}\label{fig:image}
\end{figure}
\end{CodeChunk}

\hypertarget{r-plots}{%
\subsection{R Plots}\label{r-plots}}

You can use R chunks directly to plot graphs. And you can use latex
floats in the fig.pos chunk option to have more control over the
location of your plot on the page. For more information on latex
placement specifiers see
\textbf{\href{https://en.wikibooks.org/wiki/LaTeX/Floats,_Figures_and_Captions}{here}}

\begin{CodeChunk}
\begin{figure}[H]

{\centering \includegraphics{figs/plot-1} 

}

\caption[R plot]{R plot}\label{fig:plot}
\end{figure}
\end{CodeChunk}

\hypertarget{tables}{%
\subsection{Tables}\label{tables}}

Number tables consecutively; place the table number and title (in 10
point) above the table with one line space above the caption and one
line space below it, as in Table 1. You may float tables to the top or
bottom of a column, set wide tables across both columns.

You can use the xtable function in the xtable package.

\begin{table}[H]
\centering
\begin{tabular}{rrrrr}
  \hline
 & Estimate & Std. Error & t value & Pr($>$$|$t$|$) \\ 
  \hline
(Intercept) & 0.01 & 0.12 & 0.0 & 0.97 \\ 
  x & 1.94 & 0.12 & 16.8 & 0.00 \\ 
   \hline
\end{tabular}
\caption{This table prints across one column.} 
\end{table}

\hypertarget{acknowledgements}{%
\section{Acknowledgements}\label{acknowledgements}}

Place acknowledgments (including funding information) in a section at
the end of the paper.

\hypertarget{references}{%
\section{References}\label{references}}

\setlength{\parindent}{-0.1in} 
\setlength{\leftskip}{0.125in}

\noindent

\hypertarget{refs}{}
\leavevmode\hypertarget{ref-ChalnickBillman1988a}{}%
Chalnick, A., \& Billman, D. (1988). Unsupervised learning of
correlational structure. In \emph{Proceedings of the tenth annual
conference of the cognitive science society} (pp. 510--516). Hillsdale,
NJ: Lawrence Erlbaum Associates.

\leavevmode\hypertarget{ref-NewellSimon1972a}{}%
Newell, A., \& Simon, H. A. (1972). \emph{Human problem solving}.
Englewood Cliffs, NJ: Prentice-Hall.

\bibliographystyle{apacite}


\end{document}
