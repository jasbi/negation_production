% Template for Cogsci submission with R Markdown

% Stuff changed from original Markdown PLOS Template
\documentclass[10pt, letterpaper]{article}

\usepackage{cogsci}
\usepackage{pslatex}
\usepackage{float}
\usepackage{caption}

% amsmath package, useful for mathematical formulas
\usepackage{amsmath}

% amssymb package, useful for mathematical symbols
\usepackage{amssymb}

% hyperref package, useful for hyperlinks
\usepackage{hyperref}

% graphicx package, useful for including eps and pdf graphics
% include graphics with the command \includegraphics
\usepackage{graphicx}

% Sweave(-like)
\usepackage{fancyvrb}
\DefineVerbatimEnvironment{Sinput}{Verbatim}{fontshape=sl}
\DefineVerbatimEnvironment{Soutput}{Verbatim}{}
\DefineVerbatimEnvironment{Scode}{Verbatim}{fontshape=sl}
\newenvironment{Schunk}{}{}
\DefineVerbatimEnvironment{Code}{Verbatim}{}
\DefineVerbatimEnvironment{CodeInput}{Verbatim}{fontshape=sl}
\DefineVerbatimEnvironment{CodeOutput}{Verbatim}{}
\newenvironment{CodeChunk}{}{}

% cite package, to clean up citations in the main text. Do not remove.
\usepackage{apacite}

% KM added 1/4/18 to allow control of blind submission


\usepackage{color}

% Use doublespacing - comment out for single spacing
%\usepackage{setspace}
%\doublespacing


% % Text layout
% \topmargin 0.0cm
% \oddsidemargin 0.5cm
% \evensidemargin 0.5cm
% \textwidth 16cm
% \textheight 21cm

\title{Combinatorial Capacity of English Negation in Early Child Language}


\author{{\large \bf Morton Ann Gernsbacher (MAG@Macc.Wisc.Edu)} \\ Department of Psychology, 1202 W. Johnson Street \\ Madison, WI 53706 USA \AND {\large \bf Masoud Jasbi (jasbi@ucdavis.edu)} \\ Department of Linguistics, 469 Kerr Hall, One Shields Avenue \\ Davis, CA 95 USA}


\begin{document}

\maketitle

\begin{abstract}
Negation is very important for langauge and thought. How does it develop
in the language of children? There has been many guessses like
rejection, non-existence, denail, etc. but it has been hard to assess
because these concepts are vaguge. Here we assess the combinatorial
capacity of early negation in children's productions, and use words
negation combines with as a proxy for early concepts expressed by it. We
show some important stuff.

\textbf{Keywords:}
negation; combinatorial capacity; corpus anlysis.
\end{abstract}

\hypertarget{introduction}{%
\section{Introduction}\label{introduction}}

Negation is an abstract concept, lexicalized in all previously studied
human languages, and crucial to everyday communication. It can help a
coffee shop divide its menu into ``coffee'' and ``not coffee'' sections,
with the ``not coffee'' section bringing together diverse items with no
common label. It can help us regulate each others' actions in a sign
like ``no mask, no entry''. It can also communicate our deepest wants
and dislikes, for example when we say ``I don't like Mondays''. But how
does this crucial abstract concept emerge in the human mind? Does it
emerge in its abstract and general form from the beginning or does it
develop from limited and context-specific communicative functions?

There has been several influential hypotheses on the conceptual origins
of negation and the functions it plays in early communication. Starting
a century and a half ago, Darwin (1872) thought that negation has roots
in the expression of human emotions and desires. He hypothesized that
the earliest manifestation of negation and affirmation in infants is
when they refuse food from parents, by withdrawing their heads
laterally, or when they accept the food, by inclining their heads
forward. He suggested that head shaking and nodding as common gestures
for negation and affirmation have developed from this early habit.
Similarly, many researchers studying early functions of negative
morphemes like \emph{no} proposed that children use them to ``reject''
or ``refuse'' (Bloom, 1970; Choi, 1988; Pea, 1978). For example, when
they are asked ``do you want juice?'', they may say ``no'', ``not want
it'', or ``don't like it''. Pea (1978) proposed that this function of
negation is the first to emerge in children's early language.

Bloom (1970) suggested that the use of negation to expresses
``non-existence'' emerges before rejection or refusal. For example, when
an object that children expect to be present is not present, children
may say: ``no window'', ``no fish in the bathroom'', ``Kathrine have no
socks on'' or ``I do not have underpants''. Two close concepts to
non-existence discussed in the literature are ``disappearance'' and
``non-occurrence'' (Pea, 1978; Villiers \& Villiers, 1979).
Disappearance refers to situations where an object disappears and
children use negation to express it such as ``no food. all gone'' or
``no more noise''. Non-occurrence refers to cases when an expected
action or event does not occur as in ``not working'' or ``doggie not
barking''. Some researchers referred to these cases as ``failures'' and
included examples like ``no fit in da box'' or ``it don't fit''
(Cameron-Faulkner, Lieven, \& Theakston, 2007; Choi, 1988).
Non-existence can also be expressed by negation of locative
prepositional phrases (e.g.~``no in there'' or ``daddy was not on the
phone''). While rejection was hypothesized to interact with human
emotions and desires, non-existence (broadly construed to include
``disappearance'' and ``non-occurrence'') likely interacts with human
perception. Choi (1988) proposed that children's early linguistic
negation is used to express both rejection and non-existence.

Choi (1988) introduced ``prohibition'' as a function of early negation
and suggested that it emerges as early as rejection and non-existence.
In cases of prohibition, children use linguistic negation to stop others
from performing some action. For example they may say: ``don't go'' or
``do not spill milk''. A special case of prohibition is
``self-prohibition''. For example, a child may approach prohibited food
but immediately say ``no, don't eat'' to stop themselves from doing the
prohibited action. Choi (1988) also discussed instances of negation in
which children communicate their own ``inability'' to perform an action,
for example ``I can't reach'' or ``I cannot zip it''. She suggested that
these instances emerge after the first phase (i.e.~non-existence,
rejection, prohibition). Instances of prohibition and inability are
similar in that both involve conceptualizing actions and negating them,
possibly interacting with early development of motor control.

A fourth function of negation discussed in the literature is ``denial''.
Bloom (1970) originally defined it as asserting that ``an actual or
supposed predication was not the case''. For example a child may say:
``It's not sharp''. Later researchers labeled it as ``truth-functional
negation'' and suggested that it is used to negate the truth of a
proposition (Cameron-Faulkner et al., 2007; Pea, 1978). This definition
depends on the assumed logical system and its assumptions on what type
of propositions receive truth values. In this study, we focus on a
sub-function of traditional ``denials'', namely ``labeling''. This is
often realized as the negation of nominal or adjectival predicates such
as ``this is not a bunny'', ``not red'', or ``this isn't a reptile''.
Parents often use such sentences to introduce novel linguistic labels
and facilitate word learning. Therefore, it is possible that word
learning helps the development of abstract negation. A fifth function
discussed by Choi (1988) is ``epistemic negation''. There has been no
proposal for negation originating in children's understanding of their
own or others' mental states. However, previous studies have reported
many instances of negation modifying mental state verbs such as
\emph{know}, \emph{think}, and \emph{remember} (e.g.~``I not know'').

Previous research on the origins and functions of negation has faced two
major issues. First, it has had to rely on human annotation and
classification of negative utterances in corpora which is costly,
time-consuming, and difficult. As a result, most studies have had to
focus on a handful of children and a relatively small sample of their
speech. Second, the results of previous studies have shown great
variability among children with respect to the emergence of different
functions in children's speech. For example, Nordmeyer \& Frank (2018)
looked at the speech of five children in the Providence corpus Demuth,
Culbertson, \& Alter (2006) and found a great deal of individual
variation in how early a negative function is attested, likely due to
the prevalence of different types of activities children engage in such
as book reading with parents vs.~eating food. In this paper, we address
these issues by focusing on the combinatorial and compositional capacity
of negation in children's speech. We start with the widely accepted
assumption that negation is a higher order operator, taking other
concepts as its argument. The question we ask is: what type of concepts
does linguistic negation operate on in early child language? Do we find
negation composing with a limited set of lexical items? Or do we find it
operating on a variety of lexical items and syntactic constructions?

\hypertarget{experiments}{%
\section{Experiments}\label{experiments}}

\hypertarget{data-and-preprocessing}{%
\subsection{Data and preprocessing}\label{data-and-preprocessing}}

For developmental data of child language in English, we turned to the
CHILDES database (MacWhinney, 2000), which provides child-parent
conversational
interactions.\footnote{Code and data are in quarantine at https://somewhereonearth.}.
We focused on speech produced by children with typical development
within the age range of 12 - 72 months. As this study focuses on
negation constructions at the utterance-level, we extracted individual
utterances with any of the three negation markers to be investigated in
this study: \emph{no}, \emph{not} and \emph{n't}. Utterances with only
one lexical token (e.g.~\emph{no !}) were not considered as here we aim
to address particularly the question of what negation markers could
\emph{combine} with. In addition, cases where the negation markers only
serve as a discourse marker (e.g.~\emph{no I like this one} /\emph{no no
no Mommy}) were excluded as well. Preprocessing led to a data set of
365,260 utterances with negation structures from a total of 811 children
across 56 corpora.

\begin{CodeChunk}
\begin{figure}[H]

{\centering \includegraphics{figs/speaker_stats-1} 

}

\caption[Distribution of the number of utterances with negative morphemes in child and parent speech]{Distribution of the number of utterances with negative morphemes in child and parent speech.}\label{fig:speaker_stats}
\end{figure}
\end{CodeChunk}

\hypertarget{negation-functions}{%
\subsection{Negation functions}\label{negation-functions}}

The current English data from CHILDES contains morphosyntactic
information (Sagae, Davis, Lavie, MacWhinney, \& Wintner, 2010) such as
part-of-speech (POS) information as well as grammatical or syntactic
dependency
relations.\footnote{Besides using the provided POS and syntactic dependency information in CHILDES, we also experimented with the state-of-the-art parser from Stanza [@qi-etal-2020-stanza], an open-source natural language processing library. There were no notable differences in the analyses for the negation constructions.}
We used POS tags of \emph{neg} and \emph{qn} to select negative
utterances and morphemes, the latter of which was mainly for cases with
\emph{no} as a quantifier. We further excluded cases of enumeration
(e.g.~\emph{no no no}), communicators, or discourse markers. After
extracting negative utterances, the developmental trajectories of
different constructions of interest were analyzed. Each construction was
defined to roughly match one of the communicative functions previously
discussed in the literature. In what follows we introduce each
construction and present the results. Our plots often contrast the
frequency of these constructions in children's speech as well as
parents' speech at the corresponding age of the child.

\hypertarget{rejection}{%
\subsubsection{Rejection}\label{rejection}}

For the function of ``rejection'', we examined cases where the lemma
form of the head verb of the phrase is either \emph{like} or
\emph{want}, and the head verb is modified by one of the three negative
morphemes. Each of the utterances either takes a subject or has no
subject at all. And the existence of a subject was determined via
searching for a word in the utterance that has the \emph{SUBJ}
dependency relation with the head verb.

Additionally, other than expressions that the speaker used to describe
their own emotion (e.g.~(1)) or their (in)ability to do so (e.g.~(2)),
we also included cases that express rhetorical inquiries of emotions
from one interlocutor addressed to another (e.g.~(3)) as well as
instances where the speaker is describing the emotion of somebody else
(e.g.~(4)). Overall our data extraction resulted in a total of 21,034
utterances (Child: 9,608; Parent: 11,426).

~ (1) \emph{I no like sea} / \emph{don't wanna go}

~ (2) \emph{I can't like that}

~ (3) \emph{don't you wanna try it}

~ (4) \emph{Sarah doesn't like that either}

To compare the patterns between child and parent speech, we measured the
following four metrics. The first one is the relative ratio of each of
the three negative morphemes overall. For instance, given the 9,608
utterances from child speech that serve as rejections, there are 8,531
cases with the negative morpheme \emph{no}; then the ratio of these
utterances was calculated as 8,531 / 9,608 = 0.41. The second one is the
relative ratio of negative morphemes within different head verbs
(e.g.~\emph{like} vs.~\emph{want} for rejection). For example, again
with child speech that express rejections, utterances where the negative
morphemes modify the head verb \emph{like} occur for 3,268 times; then
the ratio of these cases was computed as 3,268 / 9,608 = 0.16. The third
one is the relative ratio of the negative utterances at different ages
of the child. For instance, for rejection, at the age of 36 months, the
total number of instances with the negative morphemes in child speech is
888; then their ratio was calculated as 888 / 9,608 = 0.04.

The last one is the amount of variability in the production of the
specific function across the age span of the child, which was measured
with entropy (Cover \& Thomas, 1991) For example, after computing the
relative ratio (\emph{P(x\_i)}) of the negative utterances at a number
of \(N\) ages of the child for the specific function, the production
variability is calculated using the equation below. \[
H(X) = -\sum_{i=1}^N P(x_i)log_2P(x_i)               
\] In both child and parent speech, when articulating desires or
emotions with either of the two head verbs \emph{like} and \emph{want},
the most frequently used negative morpheme is \emph{n't} combined with
an auxiliary verb. Comparing the two different head verbs, overall the
negative morphemes co-occur with \emph{want} more frequently. With that
being said, the amount of variability in both child and parent
production is similar, a pattern that holds for both head verbs (Child
\emph{like}: 0.12; Child \emph{want} 0.12; Parent \emph{like}: 0.12;
Parent \emph{want}: 0.12).

On the other hand, when looking at the developmental trajectory, as
presented in Figure @ref(fig:emotion), children's usage of negative
morphemes is comparable regardless of the particular head verb. In
general, children start applying the negative morphemes for the function
of rejection more regularly around the age of 22 months. Within the
context of the corpus data that we analyzed, their usage of these
morphemes is the most frequent during the age range of 25 - 36 months.

\begin{CodeChunk}
\begin{figure}[H]

{\centering \includegraphics{figs/emotion-1} 

}

\caption[Rejection]{Rejection}\label{fig:emotion}
\end{figure}
\end{CodeChunk}

\hypertarget{epistemic-negation}{%
\subsubsection{Epistemic Negation}\label{epistemic-negation}}

To find epistemic uses of negation, we focused on utterances that
articulate the concept of unknowing (e.g.~(5)) or uncertainty
(e.g.~(6)). using the mental state verbs \emph{know}, \emph{remember} or
\emph{think} as the head verb, modified by the negative morphemes or the
combination of negation with auxiliaries. By these search criteria,
instances where the speaker inquires about or describes the negative
epistemic position of another speaker (e.g.~(7)) were also selected.
This led to a subset of 32,793 utterances in total (Child: 10,389;
Parent: 22,404).

~ (5) \emph{I not know} / \emph{I didn't remember}

~ (6) \emph{I don't think so}

~ (7) \emph{don't you remember} / \emph{She doesn't know this}

In both child and parent speech, the most frequently used negative
morpheme that modifies epistemic state is \emph{n't}, a pattern that is
consistent across the three different head verbs. And the negative
morphemes tend to co-occur more often in cases that describe the state
of unknowing, which is indicated mainly by the verb \emph{know}. Based
on results from Figure @ref(fig:epistemic), the production of
\emph{know} for expressions of epistemic state starts earlier in
comparison to \emph{remember} and \emph{think}. On the other hand, the
production variability for each of the head verb (\textasciitilde0.12)
is comparable to each other regardless of the particular speaker.
Overall, children began to apply the negative morphemes to articulate
this function in a more regular fashion around the age of 25 months.

\begin{CodeChunk}
\begin{figure}[H]

{\centering \includegraphics{figs/epistemic-1} 

}

\caption[Epistemic]{Epistemic}\label{fig:epistemic}
\end{figure}
\end{CodeChunk}

\hypertarget{prohibition}{%
\subsubsection{Prohibition}\label{prohibition}}

For utterances that articulate the function of prohibition, we focused
on cases where the negative morphemes are combined with the auxiliary
verb \emph{do} (\emph{do}, \emph{does}, \emph{did}) and the auxiliary
does not take any subject (e.g.~(8)). In certain cases the negative
morphemes and the auxiliary together modifies a head verb. For instances
as such, in order to not overlap with the function of rejection,
epistemic state, non-existence and possession (see below), our search
excluded cases where the head verb has any of the following lemma forms:
\emph{like}, \emph{want}, \emph{know}, \emph{think}, \emph{remember},
\emph{have}. This resulted in a total of 21,197 utterances (Child:
6,140; Parent: 15,057).

After applying our metrics, overall the most frequently used negative
morpheme is \emph{n't} when articulating prohibition. The amount of
production variability for this function is comparable in both child and
parent speech, with an approximate value of 0.12. The developmental
trajectory of using the negative morphemes to serve this function
(Figure @ref(fig:prohibition)) is comparable to that of the previous
ones, where children started more regular usage of negative morphemes
around the age of 23 months.

~ (8) \emph{don't blame Charlotte} / \emph{don't}

\begin{CodeChunk}
\begin{figure}[H]

{\centering \includegraphics{figs/prohibition-1} 

}

\caption[Prohibition]{Prohibition}\label{fig:prohibition}
\end{figure}
\end{CodeChunk}

\hypertarget{inability}{%
\subsubsection{Inability}\label{inability}}

We analyzed instances where the negative morphemes co-occur with the
auxiliary \emph{can} (\emph{can} and \emph{could}; e.g.~(9)). Again, for
instances with a head verb modified by the negative morphemes and the
auxiliary, we filtered out cases where the head verbs are the focus for
other functions. Cases that do not have a subject (\emph{can't play}) or
do not contain a subject other than I (\emph{you can't do that}) could
yield ambiguous readings without taking a larger discourse context into
account; they could be a rhetorical question or also express the concept
of prohibition. Therefore to avoid potential ambiguity, we restricted
our analyses only to cases with a subject \emph{I}. This led to a subset
of 9,150 utterances (Child: 5,410; Parent: 3,740).

~ (9) \emph{I can't see} / \emph{I can't}

Comparing child and parent production, the negative morpheme that is
used most frequently is also \emph{n't}. As shown in Figure
@ref\{fig:inaibility\} The developmental trajectory of this function is
similar to that for prohibition, and the negative morphemes are applied
more regularly starting around the age of 23 months. The amount of
production variability in both child and parent speech is approximately
0.12.

\begin{CodeChunk}
\begin{figure}[H]

{\centering \includegraphics{figs/inability-1} 

}

\caption[Inability]{Inability}\label{fig:inability}
\end{figure}
\end{CodeChunk}

\hypertarget{labeling}{%
\subsubsection{Labeling}\label{labeling}}

To capture labeling instances of denials, we concentrated on cases where
negative morphemes are adopted to indicate the identity (e.g.~(10)),
and/or characteristics (e.g.~(11)) of a predicative nominal. In
addition, we also included instances where the negative morphemes are
used to modify a predicative adjective (e.g.~(12)). Following these
criteria, utterances where the negative morpheme is modifying a nominal
or adjectival predicate of a copula verb were extracted. None of the
utterances contained expletives (\emph{there is no book}). The existence
of a predicate was identified with the help of POS information and
dependency relation. The POS of the predicate has to be either noun
(\emph{n}) or adjective (\emph{adj}), and its dependency relation with
the copula has to be \emph{PRED}. This resulted in a total of 20,329
utterances (Child: 4,793; Parent: 15,536).

~ (10) \emph{that's not a farmer}

~ (11) \emph{I'm not a heavy baby Mum}

~ (12) \emph{It's no good}

Comparing the three negative morphemes, the most frequently used is
\emph{not} regardless of the specific speaker, and the amount of
production variability is comparable (\textasciitilde0.12) between child
and parent speehc. Based on results from Figure @ref(fig:learning), the
developmental trajectory of using the negative morphemes in the domain
of language learning is comparable to previous domains. Children started
using the negative morphemes for the function of labeling nominal
objects more frequently around the age of 24 months.

\begin{CodeChunk}
\begin{figure}[H]

{\centering \includegraphics{figs/learning-1} 

}

\caption[Language learning via labeling]{Language learning via labeling}\label{fig:learning}
\end{figure}
\end{CodeChunk}

\hypertarget{non-existence}{%
\subsubsection{Non-existence}\label{non-existence}}

For the function of non-existence, we extracted utterances that either
have expletives marked by \emph{there} (e.g.~(13)), or cases where the
negative morphemes are modifying a nominal (i.e.~its syntactic head
based on the CHILDES annotation is a nominal; e.g.~(14)). With
utterances such as (14) in particular, in order to not confuse with the
function of labeling, we did not include any cases where the syntactic
head of the negative morphemes is a predicate of a copula verb
(e.g.~\emph{this is not candy}). This led to a total of 34,672
utterances (Child: 16,866; Parent: 17,806).

~ (13) \emph{there's no water}

~ (14) \emph{no (more) candy} / \emph{not your mouth}

In both child and parent speech, the most frequently occurred negative
morphemes to indicate non-existence is \emph{no}. The amount of
production variability approximates 0.12 regardless of the specific
speaker. Again for comparison of child and parent speech, we calculated
the relative ratio of (i) each of the three negative morphemes overall;
(ii) usage of negative morphemes with the two different communicative
functions; (iii) utterances expression motor control with the three
negative morphemes at different ages of the child. Overall the most
frequently used negative morpheme is \emph{n't} when applied in the
domain of motor control. Comparing the two communicative functions, the
negative morphemes tend to co-occur more often when expressing
inability. As shown in Figure @ref(fig:existence), children began
increasing their use of the negative morphemes to express non-existence
around the age of 22 months.

\begin{CodeChunk}
\begin{figure}[H]

{\centering \includegraphics{figs/existence-1} 

}

\caption[Non-existence]{Non-existence}\label{fig:existence}
\end{figure}
\end{CodeChunk}

\hypertarget{possession}{%
\subsubsection{Possession}\label{possession}}

The last function that we investigated includes utterances that are
combined with negative morphemes to denote possession. Specifically, we
selected cases where the negative morphemes are modifying a possessive
pronoun (e.g.~(15)), as well as instances where the negative morphemes
are combined with auxiliary verbs to modify a head verb with the lemma
form \emph{have} (e.g.~(16)). Again similarly to our search for
utterances that express non-existence, we excluded cases in which the
syntactic head of the negative morphemes is a predicate of a copula verb
(e.g.~\emph{this is not mine}). As a result, the total of utterances
that were subjected to analysis for this function is 9,265 (Child:
2,899; Parent: 6,366). The developmental trajectory for this function,
as shown in Figure @ref(fig:possession), is comparable to that for the
function of non-existence.

~ (15) \emph{not mine}

~ (16) \emph{I don't have it}

\begin{CodeChunk}
\begin{figure}[H]

{\centering \includegraphics{figs/possession-1} 

}

\caption[Possession]{Possession}\label{fig:possession}
\end{figure}
\end{CodeChunk}

\hypertarget{overall}{%
\subsubsection{Overall}\label{overall}}

\begin{figure*}[h]
\begin{CodeChunk}


\begin{center}\includegraphics{figs/all-1} \end{center}

\end{CodeChunk}
\caption[This image spans both columns]{All functions.}\label{fig:all}
\end{figure*}

Figure @ref(fig:all) shows the developmental trajectory for all previous
negative constructions. The y-axis is the relative frequency of each
construction relative to the frequency of all constructions within a
monthly age period. Children's negative constructions bear considerable
resemblance to parent speech in terms of the overall production
frequency. Early on, the most frequently applied function is
non-existence, while the functions with relatively smaller number of
occurrences include possession and inability. With that being said,
there are a number of observable differences between child and parent
utterances with respects to production variability. In both child and
parent production, the function that has the highest amount of
variability is non-existence (Child: ; Parent: ). There is also a
considerable difference in variability with prohibitions (Child: ;
Parent: ) and labeling (Child: ; Parent: ).

\hypertarget{discussion}{%
\section{Discussion}\label{discussion}}

Previous research on the development of negation had used
human-annotated small-scale corpus data to study early functions of
negation in children's speech. This study presented an automatic and
large-scale approach using part of speech tagging and syntactic
dependency relations to define and extract relevant constructions for
different functions of negation. We presented data on constructions
conveying rejection, prohibition, inability, none-existence, Possession,
Labeling, and epistemic states in the speech of children and adults. Our
results provide preliminary evidence for frequent use of negation in all
these constructions between 24 to 36 months of age.

We should add two important limitations of the approach presented here.
First we have used data from children's productions to assess the
development of negation as a concept. While it is possible that patterns
in children's productions reflect their comprehension and semantic
development as well, this is not guaranteed. Most importantly, there are
production-specific effects (length of utterance, ease of pronunciation,
\ldots) that we have not taken into account yet. Therefore, we can't
conclude that early emergence of some functions such as non-existence or
prohibition is necessarily conceptual. Second, given this approach's
reliance on multi-word syntactic constructions to express different
communicative functions, we miss early gestural, single-word, or
few-word expressions of these functions. To capture these cases,
traditional studies with human annoation and classification would be
more suitable.

In future work on this project, we plan to investigate the emergence of
positive counterparts to our constructions (e.g.~\emph{I don't know}
vs.~\emph{I know}). This would allow us to compare the production
trajectory of the negative constructions relative to their positive
counterpart. We plan to also focus on the developmental trajectory of
individual children to assess individual differences in the development
of negation using the methods developed in this study. Lastly, our
experiments thus far have concentrated on the utterance level, therefore
cases where negations are used as discourse markers were excluded.
However, discourse markers operating on a previous turn have important
semantic and conceptual roles in the communication between children and
parents (e.g.~Parent: \emph{do you want some bread?}; Child: \emph{no no
no}). In future work, we plan to also include such discourse level
negation to our analyses to paint a more clear and thorough picture
about the production of negation.

\hypertarget{references}{%
\section{References}\label{references}}

\setlength{\parindent}{-0.1in} 
\setlength{\leftskip}{0.125in}

\noindent

\hypertarget{refs}{}
\leavevmode\hypertarget{ref-bloom1970language}{}%
Bloom, L. M. (1970). \emph{Language development: Form and function in
emerging grammars} (PhD thesis). Columbia University.

\leavevmode\hypertarget{ref-cameron2007part}{}%
Cameron-Faulkner, T., Lieven, E., \& Theakston, A. (2007). What part of
no do children not understand? A usage-based account of multiword
negation. \emph{Journal of Child Language}, \emph{34}(2), 251.

\leavevmode\hypertarget{ref-choi1988semantic}{}%
Choi, S. (1988). The semantic development of negation: A
cross-linguistic longitudinal study. \emph{Journal of Child Language},
\emph{15}(3), 517--531.

\leavevmode\hypertarget{ref-cover_elements_1991}{}%
Cover, T. M., \& Thomas, J. A. (1991). \emph{Elements of information
theory}. New York: Wiley.

\leavevmode\hypertarget{ref-darwin1872expression}{}%
Darwin, C. (1872). \emph{The expression of the emotions in man and
animals}. John Murray.

\leavevmode\hypertarget{ref-demuth2006word}{}%
Demuth, K., Culbertson, J., \& Alter, J. (2006). Word-minimality,
epenthesis and coda licensing in the early acquisition of English.
\emph{Language and Speech}, \emph{49}(2), 137--173.

\leavevmode\hypertarget{ref-macwhinney2000childes}{}%
MacWhinney, B. (2000). \emph{The childes project: Tools for analyzing
talk. Transcription format and programs} (Vol. 1). Psychology Press.

\leavevmode\hypertarget{ref-nordmeyer2018individual}{}%
Nordmeyer, A., \& Frank, M. C. (2018). Individual variation in
children's early production of negation. In \emph{CogSci}.

\leavevmode\hypertarget{ref-pea1978}{}%
Pea, R. (1978). \emph{The development of negation in early child
language} (PhD thesis). University of Oxford.

\leavevmode\hypertarget{ref-sagae2010morphosyntactic}{}%
Sagae, K., Davis, E., Lavie, A., MacWhinney, B., \& Wintner, S. (2010).
Morphosyntactic annotation of childes transcripts. \emph{Journal of
Child Language}, \emph{37}(3), 705--729.

\leavevmode\hypertarget{ref-de1979form}{}%
Villiers, P. A. de, \& Villiers, J. G. de. (1979). Form and function in
the development of sentence negation. \emph{Papers and Reports on Child
Language Development}, \emph{17}, 57--64.

\bibliographystyle{apacite}


\end{document}
